%\begingroup
%\let\cleardoublepage\clearpage

% English abstract
%\cleardoublepage
\chapter*{Abstract}
\markboth{Abstract}{Abstract}
\addcontentsline{toc}{chapter}{Abstract (English)} % adds an entry to the table of contents
In high-throughput material design, large databases of materials are searched
for candidates with desirable characteristics. So far, searches based on
experimental data have been limited in scope, due to the vast combinatorial
space of materials, the heterogeneous quality of available data, and the
difficulty in separating the intrinsic properties of a material from those that
are contingent on the processing or the synthesis conditions. A viable
alternative is to calculate material properties using computer simulations,
that make it possible to exploit advances in parallel computing to construct
databases with millions of entries, and to obtain results that are internally
consistent. The quantitative accuracy of these predictions, however, is
dependent on the quality of the reference electronic structure calculations,
increasing the computational effort and reducing the breadth of the searches.
Data-driven approaches have been applied to reduce the cost of accurate
computational studies, by using only a small number of reference calculations
for a representative subset of materials space, and using them to train
surrogate models that predict inexpensively the outcome of such calculation on
new materials. The way materials structures are processed into a numerical
description as input of machine learning algorithms is crucial to obtain
efficient and computationally inexpensive models.  Recent advancements in the
design of information-efficient representations based on atomic densities have embedded novel types of
information, such as neighborhood environments or pair descriptions.

%
Despite the rapid development in offloading calculations to more dedicated hardware,
these enhancements nevertheless substantially increase the cost of the
representation that remains a crucial factor in simulations.  It is therefore
vital to delve deeper into the design space of representations to understand
the type of information they encapsulate.  Insights from such analyses aid in
making more informed decisions regarding the trade-off between accuracy and
performance.
While a substantial amount of work has been undertaken to compare
representations concerning their structure-property relationship, a thorough
exploration into understanding the inherent nature of the information capacity
of these representations remains mostly uncharted.
This thesis introduces a
set of measures that facilitate quantitative analysis concerning the
relationship between features and datasets, thereby assisting in such
decision-making processes and providing valuable insights to the academic community.
We demonstrate how these set of measures can be applied to analyse 
representations that are built in terms of body-order correlations of the atomic densities.
For this form of featurization we investigate the impact of different choices for the functional form, the basis functions and the induced feature space determined by the similarity measure and metric space.

Additionally, a considerable amount of effort has been dedicated to
optimize the basis set involved featurization of the  representation, typically driven by
heuristic considerations on the behavior of the regression target.  This thesis
showcases a scheme that utilizes splines to approximate the basis expansion
coefficients, paving the way for expansive optimization methods to create more
effective basis functions at no additional cost during simulation time.  This
is pivotal in simulations targeting materials encompassing a high variety of
chemical species or relying on qualitative collective variables.

Lastly, complementing these efforts is the integration of the developed methods into well-maintained and thoroughly documented packages.
This integration facilitates an enhancement of the existing methods and the incorporation of the methods into new workflows.
Specifically, this thesis introduces an implemented framework for metadynamics simulation with machine learing interatomic potentials in composition with message-passing interface of \texttt{LAMMPS}
It also explores the results obtained using this framework, particularly focusing on the finite-size effecs of the Curie point in barium titanate.
Additionally, the thesis outlines potential future developments in creating a modular machine learning ecosystem for atomistic simulations.


%Such schemes can not only be of benificial for machine learning potential but also in metadynamic simulation as.

%yields interpretative representation
%at no additional cost that
%
%optimization methods that s that
%are accurate and efficient to compu5te
%
%and computationally efficient, particularly when constructing
%representations that correspond to high-body order correlations at no
%additional cost with respect to the primitive basis by approximating it with
%splines.


%Here we take a different, unsupervised viewpoint, aiming to determine the basis
%that encodes in the most compact way possible the structural information that
%is relevant for the dataset at hand.
%For each training dataset and number of
%basis functions, one can determine a unique basis that is optimal in this sense, and can be computed at no additional cost with respect
%to the primitive basis by approximating it with splines. We demonstrate that this construction yields
%representations that are accurate and computationally efficient, particularly when constructing repre-
%sentations that correspond to high-body order correlations. We present examples that involve both
%molecular and condensed-phase machine-learning models.
%
%Over the last year I have been actively working on both these fronts. On one hand, contributing
%to the efficient implementation of state-of-the-art features for materials, based on correlation functions of the atom density. On the
%other, I have been testing the use of the Wasserstein metric, rather than the Euclidean distance, to compare such correlation functions
%- assessing the impact on the accuracy of the model, and investigating the physical meaning of the different metrics.


%% German abstract
%\begin{otherlanguage}{german}
%\cleardoublepage
%\chapter*{Zusammenfassung}
%\markboth{Zusammenfassung}{Zusammenfassung}
%% put your text here
%TODO 
%\end{otherlanguage}
%
%
%
%
%% French abstract
%\begin{otherlanguage}{french}
%\cleardoublepage
%\chapter*{Résumé}
%\markboth{Résumé}{Résumé}
%% put your text here
%TODO 
%\end{otherlanguage}


%\endgroup			
%\vfill
